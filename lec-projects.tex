\documentclass{beamer}
\usetheme{Warsaw}
\useinnertheme{circles}
\useoutertheme[subsection=false]{smoothbars}
\usepackage[utf8x]{inputenc}
\usepackage[czech]{babel}
\usepackage[T1]{fontenc}
\usepackage{listings}
\usepackage{tikz}

\begin{document}

\AtBeginSection[]
{
  \begin{frame}
    \frametitle{Outline}
    \tableofcontents[currentsection]
  \end{frame}
}

\title{Open source programování}
\subtitle{Open source projekty}
\author{Petr Baudiš $\langle${\tt pasky@ucw.cz}$\rangle$}
\institute{MFF UK 2012\\
	\vskip 1ex
	\pgfdeclareimage[height=4ex]{ccbysa}{by-sa.pdf}
	\pgfuseimage{ccbysa}
}
\date{}
\frame{\titlepage}

\section{Úvod}

\subsection{}
\begin{frame}{O čem dnes}
\begin{center}
{\bf Otevřené projekty}
\end{center}
\begin{itemize}
\item Velké projekty! Linux Kernel, Libre Office, Firefox, Apache, Perl, \dots
\item ``Obyčejné'' projekty! Git, ELinks, OpenTTD, Pachi, Bitcoin, \dots
\item Linuxové distribuce! Debian, openSUSE, \dots
\item \dots ale také Wikipedia nebo RepRap.
\vskip 4ex
\item Jaký je proces přijetí patche?
\item Kdo se stará o vydávání nových verzí projektu?
\item Jak vývojáři činí důležitá rozhodnutí?
\item Jak je projekt oficiálně zaštítěn?
\end{itemize}
\end{frame}


\section{Linux Kernel}

\subsection{}
\begin{frame}{Linux Kernel}
\begin{itemize}
\item Od roku 1991, Linus Torvalds, GPLv2
\item 15 milionů řádek kódu, $> 2000$ aktivních vývojářů; mailing list oriented
\item Maintenance: Dříve sudé/liché větve, nyní stable větve, time-based release cycle
\item Zastřešení: Linux Foundation (až od roku 2007)
\end{itemize}
\end{frame}

\subsection{}
\begin{frame}{Jikosova prezentace}
\begin{center}
	\dots
\end{center}
\end{frame}

\subsection{}
\begin{frame}{Patch Sample}
\begin{center}
% http://thread.gmane.org/gmane.linux.usb.general/55370
% http://thread.gmane.org/gmane.linux.usb.general/55406
% http://thread.gmane.org/gmane.linux.usb.general/55653
% http://thread.gmane.org/gmane.linux.usb.general/55658
% http://repo.or.cz/w/linux-2.6.git/commit/c91043adaf50ef13609003120f3471783460fb71
	\dots
\end{center}
\end{frame}


\section{Libre Office}

\subsection{}
\begin{frame}{Libre Office}
\begin{itemize}
\item Star Office $\rightarrow$ Open Office $\rightarrow$ Libre Office
\item Meritokracie, review-and-push; poslední záchrana: steering committee
\item Time-based release cycle, LGPLv3
\item Zastřešení: The Document Foundation
\end{itemize}
\end{frame}

\subsection{}
\begin{frame}{Kendyho prezentace}
\begin{center}
	\dots
\end{center}
\end{frame}


\section{GNU libc}

\subsection{}
\begin{frame}{glibc}
\begin{itemize}
\item Jeden z hlavních GNU projektů
\item TODO
\end{itemize}
\end{frame}


\section{Další projekty}

\subsection{}
\begin{frame}{Firefox}
\begin{itemize}
\item Netscape Navigator $\rightarrow$ Mozilla Suite $\rightarrow$ Firefox
\item Vývoj zejména kolem Bugzilly, meritokracie
\item Kontroverzní číslování verzí
\item Zastřešení: Mozilla Foundation
\end{itemize}
\end{frame}

\subsection{}
\begin{frame}{Apache}
\begin{itemize}
\item Původně httpd, dále různé javové knihovny, Hadoop, \dots
\item ``Konsensuální meritokracie:'' po třech měsících commiterem, +1/-1 hlasování mezi committery, -1 je veto, které lze přehlasovat 3/4 hlasů.
\item Zastřešení: Apache Software Foundation
\end{itemize}
\end{frame}

\subsection{}
\begin{frame}{Perl}
\begin{itemize}
\item Od roku 1987, Larry Wall (patch pumpkin), Perl Artistic Licence / GPL
\item Perl Artistic Licence: zverejnění změn a copyleft --- nebo přejmenování; omezení komerčního šíření
\item Patch Pumpkin; bug tracker oriented, perl5-porters
\item Zastřešení: Perl Foundation
\end{itemize}
\end{frame}

\subsection{}
\begin{frame}{Git}
\begin{itemize}
\item Od roku 2005, Linus Torvalds a pak Junio C. Hamano, GPL
\item Benevolent dictator, mailing list oriented
\item Dichotomie: linuxový kernel vs. free-cool-in web 2.0 vývojáři (github)
\item Vývoj uživatelského rozhraní "technickým způsobem"
\end{itemize}
\end{frame}

\subsection{}
\begin{frame}{OpenTTD}
\begin{itemize}
\item Přepíšeme Transport Tycoon Deluxe do C; ludde $\to$ vurlix $\to$ komunita, GPL
\item Vývoj zprvu prostřednitcvím ttdforums.net
\item Committers, bug tracker
\end{itemize}
\end{frame}

\subsection{}
\begin{frame}{ELinks}
\begin{itemize}
\item Fork Linksu od roku 2001 (links-pb $\to$ links-hacked $\to$ Experimental Links $\to$ Enhanced Links), Petr Baudiš $\to$ Jonas Fonseca $\to$ Kalle Olavi Niemitalo, GPL
\item Projekt ve spánku a bez jasného maintainera (zralý na fork?); neaktivní mailing list, bugzilla, git
\item Vývoj stylem benevolent dictator nefunguje bez diktátora
\item Nesmyslný release schedule --- realistické a {\em volné} cíle
\item Perfect is enemy of good
\end{itemize}
\end{frame}

\subsection{}
\begin{frame}{Pachi}
\begin{itemize}
\item Od roku 2007, projekt 1-2 lidí (Petr Baudiš, Jean-loup Gailly), GPL (původně část MIT)
\item Vývoj: konflikt mezi open source a snahou o prvenství ve vědeckých výsledcích a turnajích
\item Málo dobrovolníků :-( a pochybně responsivní maintainer
\item Release schedule: ``když se nám chce'' :-)
\end{itemize}
\end{frame}

\subsection{}
\begin{frame}{Bitcoin}
\begin{itemize}
\item Od roku 2007 resp. 2009, Satoshi Nakamoto $\to$ Gavin Anderson et al., GPL
\item Kdo je Satoshi Nakamoto?
\item Na vývoji závisí peníze a úspory lidí! Přitom vývoj dost nejasný\dots
\item Bitcoin Foundation --- kontroverze
\end{itemize}
\end{frame}


\section{Distribuce}

\subsection{}
\begin{frame}{Debian}
\begin{itemize}
\item Bugreporty pres Debian BTS
\item Začátek: Non-maintainer uploads (NMU), sponsoři
\item Časem se člověk stane maintainerem; archivy spravují uploadeři
\item Technické otázky: konsensus
\item Debian Project:
	\begin{itemize}
	\item Debian Social Contract, DFSG, Debian Constitution
	\item Project Leader, Secretary, Technical Committee
	\item Demokratická organizace projektu (jako Apache)
	\end{itemize}
\end{itemize}
\end{frame}

\subsection{}
\begin{frame}{OpenSUSE}
\begin{itemize}
\item Zázemí Novell / SUSE, snaha předat vývoj komunitě se moc nedaří
\item OpenSUSE Build Service
\item \dots Stickova prezentace \dots
\end{itemize}
\end{frame}


\section{Netradiční projekty}

\subsection{}
\begin{frame}{Wikipedia}
\begin{itemize}
\item Wikipedia není demokracie, vláda ``hrubého konsensu''
\item Pravidla editována společně, problémy primárně řešeny diskusí
\item Častý výskyt ``ideologických'' sporů
\vskip 4ex
\item {\em Technické} role: Správci, sysopové, byrokrati \dots \\
		accountability, předcházení revert válkám
\item {\em Technicko-politické} role: Stevardi (Wikimedia foundation)
\item {\em Politické} role (prakticky neomezené pravomoc): Arbitrážní komise, Jimmy Wales
\end{itemize}
\end{frame}

\subsection{}
\begin{frame}{RepRap}
\begin{itemize}
\item RepRap --- klasické pull requesty atd.
\item TODO
\end{itemize}
\end{frame}


\subsection{}
\begin{frame}{Děkuji za pozornost}
\begin{center}
Příště: První zápočtové prezentace! A architektura Linuxu.
\end{center}
\end{frame}

\end{document}
