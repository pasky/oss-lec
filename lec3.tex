\documentclass{beamer}
\usetheme{Warsaw}
\useinnertheme{circles}
\useoutertheme[subsection=false]{smoothbars}
\usepackage[utf8x]{inputenc}
\usepackage[czech]{babel}
\usepackage[T1]{fontenc}
\usepackage{listings}
\usepackage{tikz}

\begin{document}

\AtBeginSection[]
{
  \begin{frame}
    \frametitle{Outline}
    \tableofcontents[currentsection]
  \end{frame}
}

\title{Open source programování}
\subtitle{Otevřené vývojové prostředí}
\author{Petr Baudiš $\langle${\tt pasky@ucw.cz}$\rangle$}
\institute{MFF UK 2011\\
	\vskip 1ex
	\pgfdeclareimage[height=4ex]{ccbysa}{by-sa.pdf}
	\pgfuseimage{ccbysa}
}
\date{}
\frame{\titlepage}

\section{Úvod}

\subsection{}
\begin{frame}{Úžasný nový svět\dots}
\begin{center}
{\bf Otevřené vývojové prostředí}
\end{center}
\begin{itemize}
\item Tvořeno programátory\dots hlavně pro programátory?
\item Excelentní podpora pro systémové programátory
\item Kolísavá kvalita dokumentace, ale přístup ke zdrojákům
\item Menší důraz na IDE
\end{itemize}
\end{frame}

\subsection{}
\begin{frame}{O čem dnes}
\begin{itemize}
\item C, C++ toolchain
\item Verzovací systémy
\end{itemize}
\end{frame}


\section{C, C++ toolchain}

\subsection{}
\begin{frame}{GNU Compiler Collection}
\begin{itemize}
\item {\bf gcc}, jeden z prvních a úhelných kamenů GNU
\item Standardy --- C89, C99, C++98, C(++)1x; \\ Objective C, Fortran, Java, Ada
\end{itemize}
\begin{block}{Rozšíření}
\begin{itemize}
\item typeof, long long, {\tt x ? : y}, {\tt case 1 ... 5}, {\tt 0b1011}
\item Atributy funkcí (aj.) --- noinline, pure a const, constructor, atd.
\item Atomické operace, thread-local proměnné, vektorové typy
\item Inline assembler
\end{itemize}
\end{block}
\end{frame}

\subsection{}
\begin{frame}{Model překladu C}
\begin{itemize}
\item Preprocessing: {\tt gcc -E}
\item {\tt .c} $\to$ {\tt .o}: Překlad, {\tt gcc -c}
\item {\tt .o} $\to$ spustitelný soubor: Linkování, {\tt gcc}
\item {\tt .c} $\to$ {\tt .o}: Taky stačí {\tt gcc}
\item Jednoduché použití: {\tt gcc -Wall -O3 -g \\ -o soubor soubor.c}
\vskip 3ex
\item ELF: Univerzální formát binárních souborů \\ (objekt, executable, core dump)
\end{itemize}
\end{frame}

\subsection{}
\begin{frame}{Tvorba knihoven (shared objects)}
\begin{itemize}
\item Verzování: major.minor.patchlevel, API vs ABI, verzované symboly
\item Viditelnost symbolů: {\tt -fvisibility=hidden}, {\tt \_\_attribute\_\_((visibility ("default")))}, \\ version scripty
\item Relokace: Vyhněte se externím proměnným, uvnitř knihovny používejte lokální definice
\item Jednoduché použití: {\tt gcc -Wall -O3 -g {\bf -shared -fPIC} \\ -o soubor.so soubor.c}
\end{itemize}
\end{frame}

\subsection{}
\begin{frame}{GNU Binutils}
\begin{itemize}
\item Linker {\tt ld}: Obvykle jen backend pro gcc; linker skripty
\item Assembler {\tt as}: Vyrábí {\tt .c} z {\tt .s} místo {\tt .c} \\ AT\&T synaxe! (vs. nasm)
\item Dumpery: {\tt nm}, {\tt objdump}, {\tt readelf}
\vskip 3ex
\item Konkurence: elfutils
\end{itemize}
\end{frame}

\lstset{basicstyle=\tiny\tt}

\subsection{}
\begin{frame}[fragile]{GNU Make}
\begin{center}
Automatický překlad, přestaví právě věci, které jsou potřeba
\vskip 3ex
\begin{exampleblock}{Makefile}
\begin{lstlisting}
OBJS=soubor1.o soubor2.o

all: program
program: $(OBJS)
  $(CC) $(LDFLAGS) -o $@ $^

clean:
  rm -f $(OBJS)
\end{lstlisting}
\end{exampleblock}
{\tt make clean} \\
{\tt make} \\
\vskip 2ex
\begin{itemize}
\item Úskalí: Rekurzivní make, paralelní make, \\ automatický dependency tracking
\end{itemize}
\end{center}
\end{frame}

\subsection{}
\begin{frame}{GNU Autotools}
\begin{itemize}
\item M4: Univerzální makroprocesor
\item GNU automake: ``Zjednodušení'' tvorby Makefiles
\item GNU libtool: Portabilní výroba knihoven
\item GNU autoconf: Automatická detekce přítomných featur \\ a knihoven, compile-time konfigurace; {\tt ./configure}
\vskip 3ex
\item Buildovací návody knihoven: pkg-config
\item Konkurence: {\tt Makefile.lib} apod., cmake
\end{itemize}
\end{frame}

\subsection{}
\begin{frame}{GNU Gettext}
\begin{itemize}
\item Jak na i18n, l10n?
\item Internationalization: Systém charsetů a locales
\item Localization: Překlad textové komunikace s uživatelem
\vskip 3ex
\item Ve zdrojáku anglický string obalený makrem {\tt \_()}
\item Pro každý jazyk katalog zpráv s překlady
\item Kostra automaticky generovaná ze zdrojáku
\item Generuje separátní binární soubor, \\ runtime lookup na základě {\tt \$LC\_MESSAGES}
\item Podpora pro kontext, plurály atd.
\end{itemize}
\end{frame}

\subsection{}
\begin{frame}{GNU Debugger}
\begin{itemize}
\item {\bf gdb} --- hlavně C, ale i spousta dalších jazyků
\item Spuštění + breakpoint nebo analýza coredumpu
\item Trasování, vypisování hodnot atd.
\item Watchpointy, podmíněné breakpointy, změny hodnot, \dots
\item Interface: Rozhovor, TUI, GUI (ddd)
\item Jednoduché použití: {\tt break} a {\tt run} a {\tt cont}, {\tt bt} a {\tt frame}, \\ {\tt print} a {\tt display}, {\tt next} a {\tt step}, {\tt list}
\item Low-level: {\tt disas}, {\tt info reg} a {\tt x}, {\tt nexti} a {\tt stepi}
\vskip 3ex
\item {\tt ptrace()} syscall
\end{itemize}
\end{frame}

\subsection{}
\begin{frame}{IDE}
\begin{itemize}
\item Eclipse
\item RHIDE
\item KDevelop
\item EMACS nebo vim!
\end{itemize}
\end{frame}


\section{Verzovací systémy}

\subsection{}
\begin{frame}{Tradiční}
\begin{itemize}
\item RCS (a SCCS) --- jednotlivé soubory
\item CVS --- síťové RCS, které umí dávkově zpracovávat \\ celý adresářový strom
\item Subversion --- pořádný VCS/SCM, ale centralizovaný
\end{itemize}
\end{frame}

\subsection{}
\begin{frame}{Distribuované}
\begin{block}{Git}
It's simplest to think of the state of your Git repository as a point in a high-dimensional ``code-space'', in which branches are represented as n-dimensional membranes, mapping the spatial loci of successive commits onto the projected manifold of each cloned repository. --- \url{http://tartley.com/?p=1267}
\end{block}
\vskip 2ex
\begin{itemize}
\item Git --- nejrozšířenější(?), idiosynkratický, mocný
\item Mercurial --- přátelštější (možná)
\item Bazaar --- nejpřátelštější
\item Fossil --- vyšperkovaný, (zatím) nerozšířený
\end{itemize}
\end{frame}

\subsection{}
\begin{frame}{Děkuji za pozornost}
\begin{center}
Příště: Gitový tutorial (SU1!)
\end{center}
\end{frame}

\end{document}
