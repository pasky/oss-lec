\documentclass{beamer}
\usetheme{Warsaw}
\useinnertheme{circles}
\useoutertheme[subsection=false]{smoothbars}
\usepackage[utf8x]{inputenc}
\usepackage[czech]{babel}
\usepackage[T1]{fontenc}
\usepackage{listings}
\usepackage{tikz}

\begin{document}

\AtBeginSection[]
{
  \begin{frame}
    \frametitle{Outline}
    \tableofcontents[currentsection]
  \end{frame}
}

\title{Open source programování}
\subtitle{Otevřené vývojové prostředí}
\author{Petr Baudiš $\langle${\tt pasky@ucw.cz}$\rangle$}
\institute{MFF UK 2011\\
	\vskip 1ex
	\pgfdeclareimage[height=4ex]{ccbysa}{by-sa.pdf}
	\pgfuseimage{ccbysa}
}
\date{}
\frame{\titlepage}

\section{Úvod}

\subsection{}
\begin{frame}{Úžasný nový svět\dots}
\begin{center}
{\bf Otevřené vývojové prostředí}
\end{center}
\begin{itemize}
\item Tvořeno programátory\dots hlavně pro programátory?
\item Excelentní podpora pro systémové programátory
\item Kolísavá kvalita dokumentace, ale přístup ke zdrojákům
\item Menší důraz na IDE
\end{itemize}
\end{frame}

\subsection{}
\begin{frame}{O čem dnes}
\begin{itemize}
\item C, C++ toolchain
\item Základní knihovny
\item Dokumentace
\item Skriptovací jazyky
\item Verzovací systémy
\end{itemize}
\end{frame}


\section{C, C++ toolchain}

\subsection{}
\begin{frame}{GNU Compiler Collection}
\begin{itemize}
\item {\bf gcc}, jeden z prvních a úhelných kamenů GNU
\item Standardy --- C89, C99, C++98, C++0x TODO
\end{itemize}
\begin{block}{Rozšíření}
\begin{itemize}
\item Inline assembler
\item TODO
\end{itemize}
\end{block}
\end{frame}

\subsection{}
\begin{frame}{Model překladu C}
\begin{itemize}
\item Preprocessing: {\tt gcc -E}
\item {\tt .c} $\to$ {\tt .o}: Překlad, {\tt gcc -c}
\item {\tt .o} $\to$ spustitelný soubor: Linkování, {\tt gcc}
\item {\tt .c} $\to$ {\tt .o}: Taky stačí {\tt gcc}
\item Jednoduché použití: {\tt gcc -Wall -O3 -g \\ -o soubor soubor.c}
\vskip 3ex
\item ELF: Univerzální formát binárních souborů \\ (objekt, executable, core dump)
\end{itemize}
\end{frame}

\subsection{}
\begin{frame}{Tvorba knihoven (shared objects)}
\begin{itemize}
\item Viditelnost symbolů (TODO)
\item Relokace (TODO)
\item Export proměnných (TODO)
\item Jednoduché použití: {\tt gcc -Wall -O3 -g {\bf -shared -fPIC} \\ -o soubor.so soubor.c}
\end{itemize}
\end{frame}

\subsection{}
\begin{frame}{GNU Binutils}
\begin{itemize}
\item Linker {\tt ld}: Obvykle jen backend pro gcc; linker skripty
\item Assembler {\tt as}: Vyrábí {\tt .c} z {\tt .s} místo {\tt .c} \\ AT\&T synaxe! (vs. nasm)
\item Dumpery: {\tt nm}, {\tt objdump}, {\tt readelf}
\vskip 3ex
\item Konkurence: Gold (TODO), elfutils (TODO)
\end{itemize}
\end{frame}

\lstset{basicstyle=\tiny\tt}

\subsection{}
\begin{frame}[fragile]{GNU Make}
\begin{center}
Automatický překlad, přestaví právě věci, které jsou potřeba
\vskip 3ex
\begin{exampleblock}{Příklad}
\begin{lstlisting}
OBJS=soubor1.o soubor2.o

all: program
program: $(OBJS)
  $(CC) $(LDFLAGS) -o $@ $^

clean:
  rm -f $(OBJS)
\end{lstlisting}
\end{exampleblock}
\vskip 3ex
\begin{itemize}
\item Úskalí: Rekurzivní make, automatický dependency tracking
\item Konkurence: cmake (TODO)
\end{itemize}
\end{center}
\end{frame}

\subsection{}
\begin{frame}{GNU Autotools}
\begin{itemize}
\item M4: Univerzální makroprocesor
\item GNU automake: ``Zjednodušení'' tvorby Makefiles
\item GNU libtool: Portabilní výroba knihoven
\item GNU autoconf: Automatická detekce přítomných featur \\ a knihoven, compile-time konfigurace; {\tt ./configure}
\vskip 3ex
\item Konkurence: {\tt Makefile.lib} apod.
\end{itemize}
\end{frame}

\subsection{}
\begin{frame}{GNU Debugger}
\begin{itemize}
\item {\bf gdb} --- hlavně C, ale i spousta dalších jazyků
\item Spuštění + breakpoint nebo analýza coredumpu
\item Trasování, vypisování hodnot atd.
\item Watchpointy, podmíněné breakpointy, změny hodnot, \dots
\item Interface: Rozhovor, TUI, GUI (ddd)
\item Jednoduché použití: {\tt bt} a {\tt frame}, {\tt print} a {\tt display}, \\ {\tt next} a {\tt step}, {\tt list}
\item Low-level: {\tt disas}, {\tt info reg} a {\tt x}, {\tt nexti} a {\tt stepi}
\vskip 3ex
\item {\tt ptrace()} syscall
\end{itemize}
\end{frame}

\subsection{}
\begin{frame}{IDE}
\begin{itemize}
\item Eclipse
\item RHIDE
\item KDevelop
\item EMACS nebo vim!
\end{itemize}
\end{frame}


\section{Základní knihovny}

\subsection{}
\begin{frame}{GNU libc}
\begin{itemize}
\item Standardy, základní featury, dělení
\item Zajímavá rozšíření
\item Dynamický linker
\end{itemize}
\end{frame}

\subsection{}
\begin{frame}{Systémové knihovny}
\begin{itemize}
\item libevent
\item libnih
\item GLib
\item libucw
\end{itemize}
\end{frame}

\subsection{}
\begin{frame}{SDL}
\begin{itemize}
\item Simple DirectMedia Layer
\item Základní featury
\item Dělení
\end{itemize}
\end{frame}

\subsection{}
\begin{frame}{GTK}
\begin{itemize}
\item GIMP Toolkit
\item Základní featury
\item Architektura
\end{itemize}
\end{frame}

\subsection{}
\begin{frame}{Qt}
\begin{itemize}
\item Základní featury
\item Architektura
\end{itemize}
\end{frame}


\section{Dokumentace}

\subsection{}
\begin{frame}{\dots ostatních projektů}
\begin{itemize}
\item Manuálové stránky
\item GNU info (pinfo!)
\item Web $:-($
\end{itemize}
\end{frame}

\subsection{}
\begin{frame}{Generování dokumentace}
\begin{center}
\begin{block}{Docbook}
\begin{itemize}
\item Základní featury
\item Preprocesory (markdown, asciidoc)
\end{itemize}
\end{block}
\begin{block}{Doxygen}
\begin{itemize}
\item Z kódu
\end{itemize}
\end{block}
\end{center}
\end{frame}


\section{Skriptovací jazyky}

\subsection{}
\begin{frame}{Shell}
\begin{itemize}
\item GNU bash, zsh, (dash)
\item GNU coreutils
\item Standardy, rozšíření
\end{itemize}
\end{frame}

\subsection{}
\begin{frame}{Další}
\begin{itemize}
\item Perl
\item Python
\item Ruby, PHP, \dots
\vskip 3ex
\item SWIG
\end{itemize}
\end{frame}


\section{Verzovací systémy}

\subsection{}
\begin{frame}{Tradiční}
\begin{itemize}
\item CVS
\item Subversion
\end{itemize}
\end{frame}

\subsection{}
\begin{frame}{Distribuované}
\begin{itemize}
\item Git
\item Mercurial
\item Bazaar
\item Fossil
\end{itemize}
\end{frame}

\subsection{}
\begin{frame}{Děkuji za pozornost}
\begin{center}
Příště: Gitový tutorial (SU1!)
\end{center}
\end{frame}

\end{document}
