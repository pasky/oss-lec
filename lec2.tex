\documentclass{beamer}
\usetheme{Warsaw}
\useinnertheme{circles}
\useoutertheme[subsection=false]{smoothbars}
\usepackage[utf8x]{inputenc}
\usepackage[czech]{babel}
\usepackage[T1]{fontenc}
\usepackage{listings}
\usepackage{tikz}

\begin{document}

\AtBeginSection[]
{
  \begin{frame}
    \frametitle{Outline}
    \tableofcontents[currentsection]
  \end{frame}
}

\title{Open source programování}
\subtitle{Otevřené operační systémy}
\author{Petr Baudiš $\langle${\tt pasky@ucw.cz}$\rangle$}
\institute{MFF UK 2011\\
	\vskip 1ex
	\pgfdeclareimage[height=4ex]{ccbysa}{by-sa.pdf}
	\pgfuseimage{ccbysa}
}
\date{}
\frame{\titlepage}

\section{Úvod}

\subsection{}
\begin{frame}{Úžasný nový svět\dots}
\begin{center}
{\bf Otevřené operační systémy}
\end{center}
\begin{itemize}
\item Výborně prakticky použitelný (až na strávený čas)
\item Zdrojový kód od všech úrovní systému
\pause
\item Jádro, systémové služby, síťové servery, vývojové nástroje\\grafické rozhraní, uživatelské programy
\item Někdy i BIOS, někdy uzavřený firmware
\end{itemize}
\begin{center}
\includegraphics[height=1.6cm]{standing_daemon.jpg}\hskip 0.5em % XXX: not committed, non-free
\includegraphics[height=1.6cm]{heckert_gnu.pdf}\hskip 0.5em
\includegraphics[height=1.6cm]{Tux.pdf}\hskip 0.5em
\includegraphics[height=1.6cm]{XOrg_Logo.pdf}\hskip 0.5em
\includegraphics[height=1.6cm]{freedesktop.png}
\end{center}
\end{frame}

\subsection{}
\begin{frame}{O čem dnes}
\begin{itemize}
\item Z nabídky otevřených operačních systémů
\item Historické okénko (BSD, GNU a Linux)
\item Architektura systému Linux
\end{itemize}
\end{frame}


\section{Přehled}

\subsection{}
\begin{frame}{FreeDOS}
\begin{itemize}
\item 100\% MS-DOS compatible, GPL
\item Kompatibilita, embedded, výuka
\item Umí TCP/IP, web, přehrávání audia a videa, torrenty, \dots
\item Stabilní verze, komunita, vývoj spí
\end{itemize}
\begin{tikzpicture}[remember picture,overlay]
  \node [xshift=-4.25cm,yshift=-4.75cm,above right] at (current page.north east)
    {\includegraphics[width=4cm]{fdfish-color-text.pdf}};
\end{tikzpicture}
\end{frame}

\subsection{}
\begin{frame}{ReactOS}
\begin{itemize}
\item Windows XP/2003 compatible, GPL
\item Líbí se architektura Windows, chceme plnou kompatibilitu\\ (i s drivery atd.)
\item Alfaverze, kostra systému funguje, aplikací pramálo
\item Vývoj se sune kupředu, propojený s wine
\end{itemize}
\end{frame}

\subsection{}
\begin{frame}{Haiku}
\begin{itemize}
\item Uživatelsky přívětivý, konzistentní desktopový OS
\item Pokračování vývoje BeOS, MIT licence
\item V zásadě funkční, chybí dobrá hardwarová podpora\\ (video, wifi)
\item Návaznost: Syllable (POSIX)
\end{itemize}
\end{frame}

\subsection{}
\begin{frame}{Minix}
\begin{itemize}
\item Minimalistický POSIXový systém s mikrokernelem, BSD licence
\item Výuka, spolehlivost, embedded
\item Vývoj pomalý
\item Historicky: A. Tanenbaum, {\em Operating Systems: Design and Implementation}
\end{itemize}
\end{frame}

\subsection{}
\begin{frame}{*BSD}
\begin{itemize}
\item Berkeley Software Distribution: \\ Plynule vyvinuté z původního UNIXu
\item FreeBSD: Obecná použitelnost, výkon
\item NetBSD: Portabilita!
\item OpenBSD: Bezpečnost!
\item Součástí zdrojového stromu je kernel i základní userspace
\item Distribuce software v rámci ports
\end{itemize}
\begin{tikzpicture}[remember picture,overlay]
  \node [xshift=-4.25cm,yshift=-5.85cm,above right] at (current page.north east)
    {\includegraphics[width=4cm]{standing_daemon.jpg}};
\end{tikzpicture}
\end{frame}

\subsection{}
\begin{frame}{OpenSolaris}
\begin{itemize}
\item SVR4: Plynule vyvinuté z původního UNIXu
\item Otevření vývoje Solarisu Sunem
\item Vlastní CCDL licence
\item Téměř zánik po převzetí Oracle (OpenIndiana)
\end{itemize}
\end{frame}

\subsection{}
\begin{frame}{Linux}
\begin{itemize}
\item Nejrozšířenější otevřený OS
\item Pouze kernel, userspace různá \\ (obvykle GNU atd.)
\item Patche jdou (teoreticky) přes jediného \\ člověka (jistý Linus T.)
\end{itemize}
\begin{tikzpicture}[remember picture,overlay]
  \node [xshift=-4cm,yshift=-5.85cm,above right] at (current page.north east)
    {\includegraphics[width=3.5cm]{Tux.pdf}};
\end{tikzpicture}
\end{frame}


\section{Historie}

\subsection{}
\begin{frame}{Počátek 90. let}
\begin{itemize}
\item Většina UNIXů uzavřená, drahá (až na BSD!)
\item Na UNIX je potřeba superpočítač
\item MINIX1: 16bitový, \$69, only educational use
\item GNU: userland, překladač atd., ale chybí kernel!
\item 386BSD: portuje se na x86, právní bitva s AT\&T
\end{itemize}
\end{frame}

\subsection{}
\begin{frame}{Linus Torvalds, Helsinki}

{\tt \footnotesize
    Hello everybody out there using minix -

    I'm doing a (free) operating system (just a hobby, won't be big and professional like gnu) for 386(486) AT clones. This has been brewing since april, and is starting to get ready. I'd like any feedback on things people like/dislike in minix, as my OS resembles it somewhat (same physical layout of the file-system (due to practical reasons) among other things).

    I've currently ported bash(1.08) and gcc(1.40), and things seem to work. This implies that I'll get something practical within a few months, and I'd like to know what features most people would want. Any suggestions are welcome, but I won't promise I'll implement them :-)

    Linus (torvalds@kruuna.helsinki.fi) \\
    PS. Yes -- it's free of any minix code, and it has a multi-threaded fs. It is NOT portable (uses 386 task switching etc), and it probably never will support anything other than AT-harddisks, as that's all I have :-$($.
}

\end{frame}

\subsection{}
\begin{frame}{Raný Linux}
\begin{itemize}
\item {\em Sadly, a kernel by itself gets you nowhere. To get a working system you need a shell, compilers, a library etc. \dots Most of the tools used with linux are GNU software and are under the GNU copyleft.}
\item Tanenbaum--Torvalds debate:
\begin{itemize}
\item A: {\em \dots{}designing a monolithic kernel in 1991 is
a fundamental error.  Be thankful you are not my student.  You would not
get a high grade for such a design :-) }
\item L: {\em Your job is being a professor and researcher: That's one hell of a
good excuse for some of the brain-damages of minix. }
\item A: {\em I think it is a gross error to design an OS for any specific architecture, since that is not going to be around all that long. }
\item L: {\em An acceptable trade-off, and one that made linux possible in the first place. }
\end{itemize}
\end{itemize}
\end{frame}

\subsection{}
\begin{frame}{Distribuce}
\begin{itemize}
\item Slackware
\item Debian, Ubuntu
\item RedHat (Fedora/RHEL, CentOS, \dots)
\item SUSE (OpenSUSE/SLE)
\item Arch Linux
\item Gentoo, Linux from scratch
\end{itemize}
\end{frame}


\section{Architektura Linuxu}

\subsection{}
\begin{frame}{Jádro systému}
\begin{itemize}
\item POSIXové API, systém plně kompatibilní s UNIXem
\item Pevné ABI k userlandu, nestálé ABI v rámci jádra
\item Monolitický ale modulární, objektové C
\item Portabilní: Atmel AVR32 $\to$ IBM BlueGene
\pause
\vskip 3ex
\item Rozhraní: Systémová volání, speciální soubory, \\ speciální souborové systémy, callbacky
\end{itemize}
\end{frame}

\subsection{}
\begin{frame}{Základní userspace}
\begin{itemize}
\item util-linux --- nástroje specifické pro Linux (např. {\tt mount})
\item GNU coreutils --- základní UNIXové příkazy
\item GNU libc (glibc) --- Cčkový runtime, z API k systémovým voláním, dynamický linker
\item GNU toolchain (gcc, binutils, make)
\item Alternativy: Busybox, uClibc
\end{itemize}
\end{frame}

\subsection{}
\begin{frame}{Koordinace služeb}
\begin{itemize}
\item sysvinit / upstart + inetd, systemd
\item dbus --- message passing sběrnice
\pause
\vskip 3ex
\item {\bf Bootování:} BIOS (coreboot), GRUB, vmlinuz
\item (initrd), připojení / filesystému (read-only)
\item /sbin/init
\item Základní služby: udev, připojení souborových systémů, síť, \dots
\item Runlevel: logování, síťové služby, login manažer a obsluha tty
\end{itemize}
\end{frame}

\subsection{}
\begin{frame}{Rozhraní ``jádra''}
\begin{itemize}
\item udev --- údržba /dev souborů a spousta dalšího
\item DeviceKit: libudev (/sys), udisks, upower
\item (HAL už je naštěstí mrtev)
\item PolicyKit, ConsoleKit, PackageKit
\item NetworkManager, GStreamer / PulseAudio / ALSA, X extensions
\end{itemize}
\end{frame}

\subsection{}
\begin{frame}{Desktopové prostředí}
\begin{itemize}
\item X.org (+KMS, DRM, DRI, XI2+XRandR)
\item FreeDesktop.org
\item GNOME, KDE, Xfce, \dots
\item Firefox, SpiderMoneky, jslinux --- a jedeme znovu! ;-)
\end{itemize}
\end{frame}

\subsection{}
\begin{frame}{Skládačka}
\begin{itemize}
\item Naučte se v praxi --- Linux From Scratch!
\item Nebo alespoň Gentoo
\end{itemize}
\end{frame}

\subsection{}
\begin{frame}{Děkuji za pozornost}
\begin{center}
Příště: Zajímavé a významné open source projekty
\end{center}
\end{frame}

\end{document}
