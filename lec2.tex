\documentclass{beamer}
\usetheme{Warsaw}
\useinnertheme{circles}
\useoutertheme[subsection=false]{smoothbars}
\usepackage[utf8x]{inputenc}
\usepackage[czech]{babel}
\usepackage[T1]{fontenc}
\usepackage{listings}
\usepackage{tikz}

\begin{document}

\AtBeginSection[]
{
  \begin{frame}
    \frametitle{Outline}
    \tableofcontents[currentsection]
  \end{frame}
}

\title{Open source programování}
\subtitle{Otevřené operační systémy}
\author{Petr Baudiš $\langle${\tt pasky@ucw.cz}$\rangle$}
\institute{MFF UK 2011\\
	\vskip 1ex
	\pgfdeclareimage[height=4ex]{ccbysa}{by-sa.pdf}
	\pgfuseimage{ccbysa}
}
\date{}
\frame{\titlepage}

\section{Úvod}

\subsection{}
\begin{frame}{Úžasný nový svět\dots}
\begin{center}
{\bf Otevřené operační systémy}
\end{center}
\begin{itemize}
\item Výborně prakticky použitelný (až na strávený čas)
\item Zdrojový kód od všech úrovní systému
\pause
\item Jádro, systémové služby, síťové servery, vývojové nástroje\\grafické rozhraní, uživatelské programy
\item Někdy i BIOS, někdy uzavřený firmware
\end{itemize}
\begin{center}
\includegraphics[height=1.6cm]{standing_daemon.jpg}\hskip 0.5em % XXX: not committed, non-free
\includegraphics[height=1.6cm]{heckert_gnu.pdf}\hskip 0.5em
\includegraphics[height=1.6cm]{Tux.pdf}\hskip 0.5em
\includegraphics[height=1.6cm]{XOrg_Logo.pdf}\hskip 0.5em
\includegraphics[height=1.6cm]{freedesktop.png}
\end{center}
\end{frame}

\subsection{}
\begin{frame}{O čem dnes}
\begin{itemize}
\item Historické okénko (BSD, GNU a Linux)
\item Z nabídky otevřených operačních systémů
\item Architektura systému Linux
\end{itemize}
\end{frame}


\section{Historie}

\subsection{}
\begin{frame}{Malý návrat k historii}
\begin{itemize}
\item BSD386, GNU a Hurd, Minix a Linux
\end{itemize}
\end{frame}


\section{Přehled}

\subsection{}
\begin{frame}{FreeDOS}
\begin{itemize}
\item 100\% MS-DOS compatible, GPLv2
\item Kompatibilita, embedded vývoj, výuka
\item Umí TCP/IP, web, přehrávání audia a videa, torrenty, \dots
\item Stabilní verze, komunita, vývoj spí
\end{itemize}
\begin{tikzpicture}[remember picture,overlay]
  \node [xshift=-4.25cm,yshift=-4.75cm,above right] at (current page.north east)
    {\includegraphics[width=4cm]{fdfish-color-text.pdf}};
\end{tikzpicture}
\end{frame}

\subsection{}
\begin{frame}{ReactOS}
\begin{itemize}
\item Motivace
\item Stav
\item Vývoj
\end{itemize}
\end{frame}

\subsection{}
\begin{frame}{Haiku}
\begin{itemize}
\item Motivace
\item Stav
\item Vývoj
\end{itemize}
\end{frame}

\subsection{}
\begin{frame}{Minix}
\begin{itemize}
\item Motivace
\item Stav
\item Vývoj
\end{itemize}
\end{frame}

\subsection{}
\begin{frame}{*BSD}
\begin{itemize}
\item z UNIXu
\item FreeBSD:
\item NetBSD:
\item OpenBSD:
\item Dragonfly etc.
\item Kernel+userspace
\item Ports, CVS, commiters
\end{itemize}
\end{frame}

\subsection{}
\begin{frame}{OpenSolaris}
\begin{itemize}
\item Neutěšený stav
\end{itemize}
\end{frame}

\subsection{}
\begin{frame}{Linux}
\begin{itemize}
\item Stav
\item Vývoj jádra
\item Fragmenty userspace
\end{itemize}
\end{frame}


\section{Architektura Linuxu}

\subsection{}
\begin{frame}{Distribuce}
\begin{itemize}
\item Debian, Ubuntu
\item RedHat (Fedora/RHEL, CentOS, \dots)
\item SUSE (OpenSUSE/SLE)
\item Arch Linux
\item Gentoo, Linux from scratch
\item Slackware
\end{itemize}
\end{frame}

\subsection{}
\begin{frame}{Jádro systému}
\begin{itemize}
\item TODO
\end{itemize}
\end{frame}

\subsection{}
\begin{frame}{Dno userspace}
\begin{itemize}
\item GNU
\item glibc
\item toolchain
\item coreutils etc. GNU
\end{itemize}
\end{frame}

\subsection{}
\begin{frame}{Koordinace služeb}
\begin{itemize}
\item sysvinit
\item dbus
\item systemd
\end{itemize}
\end{frame}

\subsection{}
\begin{frame}{Rozhraní jádra}
\begin{itemize}
\item udev, udisks, upower
\item HAL R.I.P.
\end{itemize}
\end{frame}

\subsection{}
\begin{frame}{Grafické rozhraní}
\begin{itemize}
\item X.org
\item DRI, DRM, KMS
\end{itemize}
\end{frame}

\subsection{}
\begin{frame}{Desktopové prostředí}
\begin{itemize}
\item GNOME, KDE, Xfce, \dots
\item FreeDesktop.org
\item Firefox, SpiderMoneky, jslinux --- a jedeme znovu! ;-)
\end{itemize}
\end{frame}

\subsection{}
\begin{frame}{Skládačka}
\begin{itemize}
\item Naučte se v praxi --- Linux From Scratch!
\item Nebo alespoň Gentoo
\end{itemize}
\end{frame}

\subsection{}
\begin{frame}{Děkuji za pozornost}
\begin{center}
Příště: Zajímavé a významné open source projekty
\end{center}
\end{frame}

\end{document}
