\documentclass{beamer}
\usetheme{Warsaw}
\useinnertheme{circles}
\useoutertheme[subsection=false]{smoothbars}
\usepackage[utf8x]{inputenc}
\usepackage[czech]{babel}
\usepackage[T1]{fontenc}
\usepackage{listings}
\usepackage{tikz}

\begin{document}

\AtBeginSection[]
{
  \begin{frame}
    \frametitle{Outline}
    \tableofcontents[currentsection]
  \end{frame}
}

\title{Open source programování}
\subtitle{Život a správa open source projektů}
\author{Petr Baudiš $\langle${\tt pasky@ucw.cz}$\rangle$}
\institute{MFF UK 2011\\
	\vskip 1ex
	\pgfdeclareimage[height=4ex]{ccbysa}{by-sa.pdf}
	\pgfuseimage{ccbysa}
}
\date{}
\frame{\titlepage}

\section{Úvod}

\subsection{}
\begin{frame}{O čem dnes}
\begin{itemize}
\item Jak se vám daří?
%\item Zamyšlení: Richard Stallman Was Right All Along? \\
%	Aneb --- váš přístup k open source!
\vskip 3ex
\item Jak udělat svůj projekt open source
\item Co až se váš projekt stane populární
\item Když se správce a vývojář(i) neshodnou
\end{itemize}
\end{frame}


\section{Uvolněte (se), prosím}

\subsection{}
\begin{frame}{Minimální kuchařka}
\begin{itemize}
\item Měli byste mít v ruce první funkční prototyp
\item Není to žádná věda! Nechte se nést entuziasmem :-)
\item Upload na GitHub nebo jiný hosting
\item Jednoduchá homepage --- {\bf co to je}, {\bf kde to stáhnout} \\ a {\bf kam posílat patche}
\item Announcement na relevantních místech
\end{itemize}
\end{frame}

\subsection{}
\begin{frame}{Výběr licence}
\begin{itemize}
\item Jak byste to dělali?
\item Zvyková, GPL (verze?), BSD/MIT
\item Změny jsou pracné\dots (Open Street Map)
\end{itemize}
\end{frame}

\subsection{}
\begin{frame}{Neminimální kuchařka}
\begin{itemize}
\item {\bf Kašlete na dokonalý kód}
\item {\bf Kašlete na (nefatální) chyby}
\vskip 3ex
\item Čitelný kód je dobrý nápad\dots psát od začátku
\item Dokumentace: Jednostránkové README je lepší než žádné
\item Zajistěte snadný způsob postavení programu \\ (Makefile, INSTALL)
\item Infrastruktura: Nepřehánějte to!
\pause
\vskip 3ex
\item Open source podklad (vs. Eagle, BitKeeper)
\item Dobrý občan ekosystému: UNIXový nástroj! (vs. LibreOffice) \\
	Myslete na interoperabilitu (zaměnitelnost) \\ a flexibilitu (konfigurovatelnost)
\end{itemize}
\end{frame}

\subsection{}
\begin{frame}{Aby o tom někdo věděl}
\begin{itemize}
\item Není jednoduché řešení
\item Word of mouth, Google, Freshmeat
\vskip 3ex
\item Announcement --- mail nebo post
\item Tematická skupina --- uživatelé příbuzného software, \\ uživatelé stávající konkurence, zájemci o téma \dots
\item Zaujměte, ale střídmě!
\item {\bf Co to je} (v čem to je jiné), {\bf kde to stáhnout}, \\ v jakém je to stavu, na jakou platformu + v jakém jazyce \\ + pod jakou licencí to je
\end{itemize}
\end{frame}


\section{Spravujeme projekt}

\subsection{}
\begin{frame}{Metody řízení}
\begin{itemize}
\item Meritokracie! (c.f. Vzestup metirokracie)
\item Vyvíjejí {\em lidé}, ne firmy
\item The Cathedral vs. the Bazaar
\item Benevolent dictator, patch pumpkin, steering committee, committer group, ``hlasování''
\item Vyhněte se overengineeringu
\vskip 3ex
\item Právní struktura --- velké projekty, správa peněz a trademarků
\item Neziskovky, US 501(c)3
\item FSF; Linux, Apache, GNOME Foundation
\item Deštníky SFC, SPI
\end{itemize}
\end{frame}

\subsection{}
\begin{frame}{Minimální kuchařka}
\begin{itemize}
\item Odpovídejte obratem
\item Nebuďte perfekcionisty (coding style, perfect is enemy of~good)
\item Buďte perfekcionisty (vedení a idea --- dobrý vkus!)
\item Inspirujte se\dots
\vskip 3ex
\pause
\item Nabídněte co nejpřímější (psychologickou) odměnu
\item Pečlivě zdůrazňujte ostatní autory
\item Release early, release often!
\item Diskutujte, nebojte si říct o patch
\vskip 3ex
\pause
\item Vzkvétá projekt, který roste \\ (Někdo po vás fixuje bugy a udržuje projekt funkční)
\item Delegujte, nezapomínejte na user experience
\item Váš projekt by měl mít balíček!
\end{itemize}
\end{frame}

\subsection{}
\begin{frame}{Vyhněte se\dots}
\begin{itemize}
\item Problémům s větvemi a verzemi
\begin{itemize}
\item Zdržování releasů (nepřiměřené cíle)
\item Příliš košatá struktura
\item Nesmyslná čísla verzí (věčné 0.x)
\end{itemize}
\item Překážkám při stahování
\item Překážkám při přispívání
\end{itemize}
\end{frame}

\subsection{}
\begin{frame}{Uživím se tím?}
\begin{itemize}
\item Není jednoduché řešení
\vskip 3ex
\item Placená podpora (Cygnus, Red Hat, Novell, Oracle, \dots)
%  Cygnus 'Open Sources'
\item Placená rozšíření (Novell, CMS, volná noha)
\item Double licencing (Qt, MySQL)
\item Reklama (Mozilla, BackTrack, MySQL)
\item Dary (Wikipedia, někdy bug bounties)
\item HW příslušenství (embedded)
\item Infrastruktura (IBM)
\end{itemize}
\end{frame}

\subsection{}
\begin{frame}{A jinak?}
\begin{itemize}
\item Volnočasová aktivita (ale proč to lidé dělají?)
\pause
\begin{itemize}
\item ``Scratching an itch''
\item Konstruktivní prokrastinace --- závislost
\item Puntík v CVčku nebo world domination
\item {\bf Ego-boo!} (vs. self-deprecation)
\end{itemize}
\pause
\item Google Summer of Code
\item Zaměstnání v OSS společnosti
\item Zaměstnání v Google atd.
\item Nečekané výhody
\end{itemize}
\end{frame}


\section{Fork me!}

\subsection{}
\begin{frame}{Proč}
\begin{itemize}
\item Blbej maintainer (Xorg, LibreOffice)
\item Neaktivní vývoj (kompoZer, dbndns, ELinks)
\item Pokračování svobodné verze (MariaDB, Xonotic)
\item Jiná sada uživatelů, technické změny (WebKit, mplayer2)
\item Vývoj vs. kompatibilita (GNU Arch --- Bazaar --- Bazaar NG --- Baz)
\item Jiná sada featur (eglibc, kernelové větve)
\item ``Vývojová větev'' (egcs, kernelové větve)
\vskip 3ex
\item Zkuste nejdříve jinou cestu, ale nebojte se!
\item Fork zanikne (nejčastěji), připojí se zpět, nahradí originál \\ nebo žije paralelně
\end{itemize}
\end{frame}

\subsection{}
\begin{frame}{Jak}
\begin{itemize}
\item Publikuji svojí verzi
\item (Nenásilně) upozorním ostatní (vývojáře, uživatele)
\item Upstream/downstream mergování
\vskip 3ex
\item Pozor na kompatibilitu API!
\end{itemize}
\end{frame}


\subsection{}
\begin{frame}{Děkuji za pozornost}
\begin{center}
Rekapitulace:
\begin{itemize}
\item Opensourcovat je snadné, neodkládejte to!
\item Vyhněte se overengineeringu ve všech směrech
\item Myslete na motivaci --- ego-boo
\end{itemize}
\vskip 3ex
Příště: Závěrečná show (ve vaší režii).
\end{center}
\end{frame}

\end{document}
